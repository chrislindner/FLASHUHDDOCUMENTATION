\documentclass[preprint,11pt]{aastex}
%\documentclass{emulateapj}
\pdfoutput=1
\usepackage{apjfonts,amsmath}
\usepackage{graphicx,verbatim}
\newcommand{\beq}{\begin{equation}}
\newcommand{\eeq}{\end{equation}}
\newcommand{\bea}{\begin{eqnarray}}
\newcommand{\eea}{\end{eqnarray}}

\begin{document}	

\title{FLASH UHD Documentation}
\author{
Christopher C. Lindner,
Milo\v s Milosavljevi\'c,
}
\affil{Department of Astronomy, University of Texas, 1 University Station C1400, Austin, TX 78712}

\righthead{SIMULATIONS OF RADIATION DRIVEN OUTFLOWS}
\lefthead{LINDNER ET AL.}

\begin{abstract}
The purpose of this documentation is to document the FLASH unsplit hydrodynamics solver and the changes needed to add radiation hydrodynamics to this solver.

\keywords{ }


\end{abstract}

\section{Introduction}
\label{sec:intro}
\setcounter{footnote}{0}

The latest version of the FLASH code is available in my directory \\
\verb!/data1/r900-1/lindner/FLASH4.2.1/source/! \\

Most of the files relevant to the unsplit solver are in \\
\verb!/data1/r900-1/lindner/FLASH4.2.1/source/physics/Hydro/HydroMain/unsplit/Hydro_Unsplit!

Any file or directory names I mention will be relative to this directory.

Section \ref{sec:flowchart} contains a basic overview flowchart of how the solver works.  There are some brief descriptions of select files as well.

\section{General Flowchart}
\label{sec:flowchart}
\begin{description}
	\item[Hydro] Main FLASH hydro driver that calls the unsplit solver.
	\item[hy\_uhd\_unsplit] This basically runs through the entire unsplit solver.  It calls the following functions (\verb!hy_uhd_! prefixes have been ommitted)
	\begin{description}
		\item[putGravityUnsplit]
		\item[getRiemannState]  Calculates and stores Riemann state values at cell facethem so we can use these to compute fluxes, e.g. MC 4.2.3
		\begin{itemize}
			\item First, some "`hybrid order"' things that we do not use (?)
			\item Then, slope flattening is carried out: MC 4.2.2
			\item Then, we start calculating Riemann states
			\item \textbf{dataReconstOneStep}
			\item Applies geometric terms
			\item Updates Gravity
			\item Stores scratch terms for each direction
			\item Applies transverse correction terms for 3D
			\item \textbf{upwindTransverseFlux}
		\end{itemize}
		\item[getFaceFlux]
		\item[unsplitUpdate]
		\item[unsplitUpdateMultiTemp]
		\item[energyFix]
		\item[Grid\_conserveFluxes]
		\item[Eos\_wrapped]
		\item[putGravityUnsplit]
		\item[addGravityUnsplit]
		\item[energyFix]
		\item[multiTempAfter]

	
	\end{description}
	
\end{description}
\end{document}